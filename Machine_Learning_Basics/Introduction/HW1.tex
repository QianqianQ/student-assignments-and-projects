% *** DOCUMENT TYPE ***
\documentclass[article,11pt]{article}

% *** GRAPHICS RELATED PACKAGES ***
\usepackage{caption}
\usepackage{graphicx}
\graphicspath{{resources/}} % path where graphic files are
\DeclareGraphicsExtensions{.pdf,.jpeg,.png} % graphic files extensions

% *** LENGUAGE RELATED PACKAGES ***
\usepackage[english]{babel}
\usepackage[utf8]{inputenc}

% *** MATH RELATED PACKAGES ***
\usepackage{amsmath}
\usepackage{bm, amssymb}

% *** OTHER PACKAGES ***
\usepackage{fullpage}
\usepackage{url}

% *** COMMANDS AND MACROS ***
\renewcommand*{\thesection}{Problem~\arabic{section}:}
\renewcommand{\labelenumi}{(\alph{enumi})}
\newcommand{\Tt}[0]{\boldsymbol{\theta}}
\newcommand{\XX}[0]{{\cal X}}
\newcommand{\ZZ}[0]{{\cal Z}}
\newcommand{\vx}[0]{{\bf x}}
\newcommand{\vv}[0]{{\bf v}}
\newcommand{\vu}[0]{{\bf u}}
\newcommand{\vs}[0]{{\bf s}}
\newcommand{\vm}[0]{{\bf m}}
\newcommand{\mX}[0]{{\bf X}}
\newcommand{\vw}[0]{{\bf w}}
\newcommand{\D}[0]{{\mathcal{D}}}
\newcommand{\mP}{\mathbf{P}}
\newcommand{\E}[0]{{\mathbb{E}}}
\newcommand{\vy}[0]{{\bf y}}
\newcommand{\vb}[0]{{\bf b}}
\newcommand{\vr}[0]{{\bf r}}
\newcommand{\vz}[0]{{\bf z}}
\newcommand{\N}[0]{{\mathcal{N}}}
\newcommand{\vc}[0]{{\bf c}}
\newcommand{\norm}[1]{\Vert  {#1} \Vert}
%\newcommand{\XX}[0]{X}
%\newcommand{\ZZ}[0]{Z}
\newcommand{\bmx}[0]{\begin{bmatrix}}
\newcommand{\emx}[0]{\end{bmatrix}}
%\newcommand{\norm}[1]{\lVert#1\rVert}
\newcommand{\T}[0]{\text{T}}
\DeclareMathOperator*{\rank}{rank}
\newcommand\defeq{:=}
\newcommand\eigvecS{\hat{\mathbf{u}}} 
\newcommand\eigvecCov{\mathbf{u}} 
\newcommand{\featuredim}{d}
\newcommand{\samplesize}{N}
\newcommand{\sampleidx}{i} 
\newcommand{\target}{y}
\newcommand{\error}{E}
\newcommand\dataset{\mathbb{X}}
\newcommand{\inspace}{\mathcal{X}}
\newcommand{\sigmoid}{\sigma}
\newcommand{\outspace}{\mathcal{Y}}
\newcommand{\hypospace}{\mathcal{H}}
\newcommand{\emperror}{\mathcal{E}}
\newcommand{\generror}{\mathcal{E}}
\DeclareMathOperator{\supp}{supp}
%\newcommand{\loss}[3]{L({#1},{#2},{#3})} 
\newcommand{\loss}[2]{L({#1},{#2})} 
\newcommand{\determinant}[1]{{\rm det}({#1})} 
\newcommand{\prob}[1]{P({#1})} 
\newcommand{\pdf}[1]{p({#1})} 
\def \expect {\mathsf{E} }
\DeclareMathOperator*{\argmax}{argmax}
\DeclareMathOperator*{\argmin}{argmin}
%\def \prob {{\rm P} }





% *** DOCUMENT TITLE ***
\title{CS-E3210- Machine Learning Basic Principles \\ Home Assignment 1 - ``Introduction''}
\begin{document}
\date{}
\maketitle

Your solutions to the following problems should be submitted as one single pdf which does not contain 
any personal information (student ID or name). The only rule for the layout of your submission is that 
each problem has to correspond to one single page, which has to include the problem statement on top 
and your solution below. You are welcome to use the \LaTeX-file underlying this pdf, 
available under \url{https://version.aalto.fi/gitlab/junga1/MLBP2017Public}, and fill in your solutions there. 



% *** PROBLEM 1 ***
\newpage
\section{Let The Data Speak - I}
In the folder ``Webcam'' at \url{https://version.aalto.fi/gitlab/junga1/MLBP2017Public} you will find $\samplesize=7$  
webcam snapshots $\vz^{(1)},\ldots,\vz^{(\samplesize)}$ with filename ``shot??.jpg''. Import these snapshots into your favourite 
programming environment (Matlab, Python, etc.) and determine for each snapshot $\vz^{(\sampleidx)}$ its greenness $x_{g}^{(\sampleidx)}$ 
and redness $x_{r}^{(\sampleidx)}$ by summing the green and red intensities over all image pixels (cf.\ \url{https://en.wikipedia.org/wiki/RGB_color_model}). 
Produce a scatter plot (cf.\ \url{https://en.wikipedia.org/wiki/Scatter_plot}) with the points $\vx^{(\sampleidx)}=(x^{(\sampleidx)}_{r},x^{(\sampleidx)}_{g})^{T} \in \mathbb{R}^{2}$, for $i=1,\ldots,\samplesize$. 
Do not forget to label the axes of your plot. 
%wher
%with $x$-axis representing the redness and $y$-axis representing the greenness. 

\noindent{\bf Answer.}

\begin{table}[h]
\centering
\begin{tabular}{|c|c|c|}
\hline
\textbf{Images} & \textbf{Red} & \textbf{Green} \\ \hline\hline
shot1           & 65069350       & 65398449        \\ \hline
shot2           & 64797631       & 64619047         \\ \hline
shot3           & 62725984       & 63548262         \\ \hline
shot4           & 58760119       & 58666604         \\ \hline
shot5           & 58960696       & 57416785         \\ \hline
shot6           & 60470606       & 61338129         \\ \hline
shot7           & 60728760      & 59001882         \\ \hline
\end{tabular}
\caption{Green and red pixel intensities for each image.}
\label{tab:t1}
\end{table}
 \ref{fig:cats1}.

\vspace*{-1ex}

\begin{figure}[h!]
\begin{center}
\includegraphics[height=8.8cm]{1_1.jpeg}
\end{center}
\caption{Scatter plot - Redness vs Greenness}
\label{fig:cats1}
\end{figure}


% *** PROBLEM 2 ***
\newpage
\section{Let The Data Speak - II}
Familiarize yourself with random number generation in your favourite 
programming environment (Matlab, Python, etc.). In particular, try to generate a data set 
$\{ \vz^{(\sampleidx)} \}_{i=1}^{\samplesize}$ containing $\samplesize=100$ vectors $\vz^{(\sampleidx)} \in \mathbb{R}^{10}$, which 
are drawn from (i.i.d. realizations of) a Gaussian distribution $\mathcal{N}(\mathbf{0},\mathbf{I})$ with zero mean and covariance 
matrix being the identity matrix $\mathbf{I}$. For each data point $\vz^{(\sampleidx)}$, compute the two 
features 
\begin{equation}
x_{1}^{(\sampleidx)} = \mathbf{u}^{T} \vz^{(\sampleidx)} \mbox{, and } x_{2}^{(\sampleidx)}= \mathbf{v}^{T} \vz^{(\sampleidx)}, 
\end{equation}
with the vectors $\mathbf{u}\!=\!(1,0,\ldots,0)^{T}\in \mathbb{R}^{10}$ and $\mathbf{v}\!=\!(9/10,1/10,0,\ldots,0)^{T}\in \mathbb{R}^{10}$. 
Produce a scatter plot (cf.\ \url{https://en.wikipedia.org/wiki/Scatter_plot}) with the points 
$\vx^{(\sampleidx)}=(x^{(\sampleidx)}_{1},x^{(\sampleidx)}_{2})^{T} \in \mathbb{R}^{2}$, for $\sampleidx=1,\ldots,\samplesize$. 
Do not forget to label the axes of your plot. 

\noindent{\bf Answer.}
 \ref{fig:cats2}.

\vspace*{-1ex}

\begin{figure}[h!]
\begin{center}
\includegraphics[height=8.8cm]{1_2.jpeg}
\end{center}
\caption{Scatter plot - $x_{1}$ vs $x_{2}$}
\label{fig:cats2}
\end{figure}


% *** PROBLEM 3 ***
\newpage
\section{Statistician's Viewpoint}

Consider you are provided a spreadsheet whose rows contain the data points $\vz^{(\sampleidx)}=(\sampleidx,y^{(\sampleidx)})$, with row index $\sampleidx=1,\ldots,\samplesize$.  
A statistician might be interested in studying how to model the data using a probabilistic model, e.g., 
\begin{equation} 
y^{(\sampleidx)} = \mu + \sigma e^{(\sampleidx)} 
\end{equation}
where $e^{(\sampleidx)}$ are i.i.d. standard normal random variables, i.e., $e^{(\sampleidx)} \sim \mathcal{N}(0,1)$.
\begin{itemize}
\item Which choice for $\mu$ best fits the observed data?
\item Given the optimum choice for $\mu$, what would be the best guess for $y^{(\samplesize+1)}$?
\item Can we somehow quantify the uncertainty in this prediction?
\end{itemize}

\noindent {\bf Answer.}

The $\mu$ value that would best fit the observed data is: \\
\indent{Based on the Maximum Likelihood Estimation method,} \\ 
\begin{center}
$y^{(i)} \sim \mathcal{N}(\mu,\sigma^{2})$ \\
$\mu_{{ML}}=arg\mathop{max}\limits_{\mu}p(Y \mid \mu) = arg\mathop{max}\limits_{\mu}\prod\limits_{i=1}^{N}p(y^{(i)}\mid \mu)$\\
$\frac{dp(Y \mid \mu)}{d \mu}=0$
\end{center}

\begin{equation}
	\mu = \bar{y}= \frac{\sum\limits_{i=1}^{N}y^{(i)}}{N}
\end{equation}

\indent{The best prediction for $y^{(\samplesize+1)}$ would be:} \\
\indent{Based on knowledge about posterior predivtive distributions,} 
\begin{equation}
y^{(\samplesize+1)} = \bar{y}= \frac{\sum\limits_{i=1}^{N}y^{(i)}}{N}
\end{equation}

The uncertainty of the prediction can be quantified as: \\
\indent{Based on the Mean Square Error and Maximum Likelihood Estimation methods,} \\
\indent{and we know that $\bar{y}$ is an unbiased estimator,}
\begin{equation}
\begin{aligned}
MSE = & Bias(\bar{y})^{2}+Var(\bar{y}) \\
	= & 0 + \frac{\sigma^{2}}{N} \\
	= & \frac{\sigma^{2}}{N}\\
	SE_{\bar{y}} = \frac{\sigma}{\sqrt{N}}\\
\end{aligned}
\end{equation}

% *** PROBLEM 4 ***
\newpage
\section{Three Random Variables}
Consider the following table which indicates the presence of a particular property 
(`A', `B' or `C') for a number of items (each item corresponds to one row). 
% where each row corresponds to the presence of 
%some property for a particular item. You can assume that the items are independent. 

\begin{tabular}{c|c|c}
\hline\hline
A & B & C \\ [0.5ex] % inserts table %heading
\hline
1 & 0 & 1\\
1 & 1 & 0 \\
1 & 0 & 1 \\
1 & 1 & 0 \\
\hline
\end{tabular}

\begin{itemize}
\item Can we predict if an item has property `B' if we know the presence of property `C' ?
\item Can we predict if an item has property `A' if we know the presence of property `C' ?
\end{itemize} 

\noindent {\bf Answer.} \\ \\
$P(\bar{B} \mid C)=1,P(B \mid C)=0, P(\bar{B} \mid C)+P(B \mid C)=1$ \\
$P(B \mid \bar{C})=1, P(\bar{B} \mid \bar{C})=0, P(B \mid \bar{C})+P(\bar{B} \mid \bar{C})=1$ \\
We can predict if an item has property `B' if we know the presence of property `C'. If an item has property 'C', we can predict that it does not have property 'B'; and if it does not have 'C', it will have 'B'.\\ \\
$P(\bar{A} \mid C)=0, P(A\mid C)=1,P(\bar{A} \mid C)+P(A\mid C)=1$\\
$P(A \mid \bar{C})=1, P(\bar{A}\mid \bar{C})=0,P(A \mid \bar{C})+P(\bar{A}\mid \bar{C})=1$\\
We can predict if an item has property `A' if we know the presence of property `C'. The item samples show that an item always has property 'A', no matter whether it has 'C' or not.\\ \\

\noindent{(If the size of samples is big enough, we can use fisher's exact test (or chi-square test) to do correspondence analysis.)}

\begin{figure}[!h]
	\centering
	\begin{minipage}{.45\textwidth}
		\centering
		\hspace*{0em}\includegraphics[width=1.1\linewidth]{1_4_1.jpeg}
		\caption{Stacked bar C-B}
		\label{fig:cats3}
	\end{minipage}\hspace{1cm}
	\begin{minipage}{.45\textwidth}
		\centering
		\hspace*{0em}\includegraphics[width=1.1\linewidth]{1_4_2.jpeg}
		\caption{Stacked bar C-A}
		\label{fig:cats4}
	\end{minipage}
\end{figure}
% *** PROBLEM 5 ***
\newpage
\section{Expectations}
Consider a $d$-dimensional Gaussian random vector $\vx \sim \mathcal{N}(\mathbf{0},\mathbf{I})$, 
a random variable $e \sim \mathcal{N}(0,\sigma^{2})$. 
For a fixed (non-random) vector $\mathbf{w}_{0} \in \mathbb{R}^{d}$, we construct the random variable $y= \vw_{0}^{T} \vx + e$. 
Now consider another arbitrary (non-random) vector $\mathbf{w} \in \mathbb{R}^{d}$. 
Find a closed-form expression for the expectation $\expect [ ( y - \vw^{T} \vx )^{2} ]$ in terms of the variance $\sigma^{2}$ and the vectors $\vw, \vw_{0}$. 

\noindent {\bf Answer.}

\begin{align*}
\expect \{( y - \vw^{T} \vx )^{2} \} & =\expect \{ ((\vw_{0}^{T}-\vw^{T})\vx+e) * ((\vw_{0}^{T}-\vw^{T})\vx+e)\} \\
& = \expect \{ (\vw_{0}^{T}-\vw^{T})^{2}\vx^{2} + 2e(\vw_{0}^{T}-\vw^{T})\vx+e^{2} \} \\
& = (\vw_{0}^{T}-\vw^{T})^{2} \expect [\vx^{2}] + 2(\vw_{0}^{T}-\vw^{T})\expect[e\vx]+\expect[e^{2}]
\end{align*}

\begin{center}
	$\expect[\vx^{2}]=Var(\vx)+\expect[\vx]^{2} = [1,1,...1]^{T} \in \mathbb{R}^{d}$ \\
	$\expect[e\vx]=\frac{\mu_{e}\sigma_{\vx}^2+\mu_{\vx}\sigma_{e}^2}{\sigma_{e}^{2}+\sigma_{\vx}^{2}}=0 $ \\
	$\expect[e^{2}]=\expect[(e-\expect[e])*(e-\expect[e])]=\sigma^{2}$
\end{center}

\begin{align*}
	\expect \{( y - \vw^{T} \vx )^{2} \}&=(\vw_{0}^{T}-\vw^{T})^{2}[1,1,...1]^{T}+\sigma^{2} \\
	  &=(\vw_{0}-\vw)^{T}(\vw_{0}-\vw)+\sigma^{2}
\end{align*}
\end{document}

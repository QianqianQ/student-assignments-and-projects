\documentclass{article} 
\usepackage{listingsutf8}
\usepackage{float}
\usepackage{caption}
\usepackage{graphicx}
\usepackage{amsmath}
\usepackage{amssymb}

  \title{\huge Bayesian Data Analysis - Assignment 1} 
\begin{document} 
  \maketitle 
  \section{Basic probability theory and terms} 
    \subsection*{a)} 
      \paragraph*{probability}is the measure of the likelihood of a given event's occurrence, which is expressed as a number between 0 and 1. 
      \paragraph*{probability mass}is a function that gives the probability that a discrete random variable is exactly equal to some value. ($f_{X}(x)=P(X=x)=P(\{s \in S:X(s)=x\})$)
      \paragraph*{probability density}is a function of a continuous variable whose integral over a region gives the probability that a random variable falls within the region. ($P(a \leqslant X \leqslant b)= \int_a^b f_{X}(x) \,dx$)
      \paragraph*{probability mass function (pmf)}is a function that gives the probability that a discrete random variable is exactly equal to some value.
      \paragraph*{probability density function (pdf)}is a function of a continuous variable whose integral over a region gives the probability that a random variable falls within the region.
      \paragraph*{probability distribution}is a function that provides the possibilities of occurrence of all the different possible values (events).
      \paragraph*{discrete probability distribution} is a table (or a formula) listing all possible values that a discrete variable can take on, together with the associated probabilities.
      \paragraph*{continuous probability distribution}describes the probabilities of the possible values of a continuous random variable. 
      \paragraph*{cumulative distribution function (cdf)}is a function that gives probability that random variable is less than or equal to a value. ($F_{X}(x) =P(X \leqslant x) $) 
  	\subsection*{b)}
  	 \paragraph*{sampling distribution}is the probability distribution of sample statistics based on randomly selected samples from the same population. 
  	 \paragraph*{observation model}is a mathematical model (probability distribution) that relates the parameters of the model to the observations.
  	 \paragraph*{statistical model}is a class of mathematical model (probability distribution) on sample space, which embodies a set of assumptions concerning the generation of some sample data, and similar data from a larger population.  
  	 \paragraph*{likelihood}is a function of the parameters of a statistical model given data, which is equal to the probability (density) assumed for those observed outcomes given those parameter values. ($L(\theta \mid x)=P(x \mid \theta)$)
  	 
  \section{Basic computer skills} 
   The language used is Python. The source code is attached in the appendix.
    \subsection*{a)}
     \begin{figure}[H]
    	\centering
    	\captionsetup{justification=centering}
    	\includegraphics[width=10cm]{a.jpeg}
        \caption{density function}
     \end{figure}
  
 	\subsection*{b)}
 	\begin{figure}[H]
 		\centering
 		\captionsetup{justification=centering}
 		\includegraphics[width=10cm]{b.jpeg}
 		\caption{histogram of the samples}
 	\end{figure}
 	We can see that its curve trend is very similar to that of the density function.
 	\subsection*{c)}
 		\begin{table}[H]	
 		\centering
 		\begin{tabular}{r|c|c}
 		 \hline	
 	\quad  & mean    & variance \\
 	\hline
 	sample  & 0.199498  & 0.009546 \\
 	true    & 0.200000  & 0.010000 \\
 	\hline
 	  \end{tabular}
	 \end{table}
    From the table, we can see that the sample mean and variance match (roughly) to the true mean and variance of the distribution.
 	\subsection*{d)}	
 		\begin{center}
		The central 95 \%-intercal: [ 0.05141249  0.41952854]
		\end{center}
	
  \section{Bayes' theorem}
	Assume that A is the event that having lung cancer, B is the event that test gives a positive result.
	\begin{gather*}
		P(A)= 0.001\\
		P(B \mid A)= 0.98 \\
		P(B \mid \bar{A})= \bar{P}(\bar{B} \mid \bar{A})=0.04 \\
	\end{gather*}
	\begin{equation*}
	 \begin{aligned}
	 P(B)&=P(B \mid A)P(A)+P(B \mid \bar{A})P(\bar{A})\\
	 	 &=0.04094
	 \end{aligned}
	\end{equation*}
	If the test gives a positive result, the probability of having lung cancer:
	\begin{equation*}
	 \begin{aligned}
		P(A \mid B)&= \frac{P(B \mid A)P(A)}{P(B)} \\
				   &= \frac{49}{2047} \approx 2.4\%
	 \end{aligned}
	\end{equation*}	
	There is only a possibility of 2.4\% having lung cancer when the test gives a positive result. So I think it is not a good idea to introduce the test to market.
	(The claimed 97\% successful rate maybe comes from a unreasonable ratio of healthy testees to testees having lung cancer. A relatively high proportion of testees having lung cancer will cause higher succeessful rate. However, we should take into account that lung cancer is rare in general population)
  \section{Bayes' theorem}
   \paragraph{}The probabilities of selecting box A, B, C:
    \begin{gather*}
      P(A)=\frac{2}{5} \\
      P(B)=\frac{1}{10} \\
      P(C)=\frac{1}{2}
    \end{gather*}
   \paragraph{}The probabilities of picking up a red ball from box A, B, C separately:
   \begin{gather*}
  		P(red \mid A)=\frac{2}{7} \\
  		P(red \mid B)=\frac{4}{5} \\
  		P(red \mid C)=\frac{1}{4} 
  	\end{gather*}
   \paragraph{}The probabilities of picking up a red ball:
    \begin{equation*}
     \begin{aligned}
     P(red)&=P(A)P(red \mid A)+P(B)P(red \mid B)+P(C)P(red \mid C)\\
     	   &=\frac{2}{5} \times \frac{2}{7} + \frac{1}{10} \times \frac{4}{5} + \frac{1}{2} \times \frac{1}{4}\\
     	   &= \frac{447}{1400} \approx 31.9\%
     \end{aligned}
    \end{equation*}
   \paragraph{}If a red ball is picked up, the probabilities of picking up from box A, B, C separately:
   \begin{equation*}
  	 \begin{aligned}
  			P(A \mid red)&=\frac{P(red \mid A)P(A)}{P(red)}\\
  						&= \frac{160}{447}
  	 \end{aligned} 		
  	\end{equation*}
   \begin{equation*}
  	\begin{aligned}
  			P(B \mid red)&=\frac{P(red \mid B)P(B)}{P(red)}\\
						&= \frac{112}{447}
  	\end{aligned} 		
   \end{equation*}
   \begin{equation*}
  	\begin{aligned}
  			P(C \mid red)&=\frac{P(red \mid C)P(C)}{P(red)}\\
  						&= \frac{175}{447}
  	\end{aligned} 		
   \end{equation*}
   \paragraph{}So it mostly came from box C.
  
  \section{Bayes' theorem}	
  	\begin{gather*}
  		P(fraternal \quad twins)= \frac{1}{125} \\
  		P(identical \quad twins)= \frac{1}{300} \\
  		P(males)=P(females)= \frac{1}{2}\\
  		P(male \quad twins \mid fraternal \quad twins)= \frac{1}{4} \\
  		P(male \quad twins \mid identical \quad twins)= \frac{1}{2}
  	\end{gather*}
  	\paragraph{}The probability of birth of male twins:
  	 \begin{equation*}
  	  \begin{aligned}
  	 	P(male \quad twins)&=P(fraternal \quad twins)P(male \quad twins \mid fraternal \quad twins)\\
  	 	& \quad +P(identical \quad twins)P(male \quad twins \mid identical \quad twins) \\
  	 	&= \frac{11}{3000}
  	  \end{aligned}
	 \end{equation*}
	\paragraph{}The probability that the male twins are identical twins:
	\begin{equation*}
	 \begin{aligned}
	 &P(identical \quad twins \mid male \quad twins)\\
	 =&\frac{P(male \quad twins \mid identical \quad twins)P(identical \quad twins)}{P(male \quad twins)}\\
	 		=& \frac{5}{11} \approx 45.5\%
	 \end{aligned}		
	\end{equation*}

 \clearpage
 \appendix
 \leftline{\LARGE \textbf{Appendix}}
 \section* {{\large Source code}}
	\begin{lstlisting}[language=python]
	import numpy as np
	from scipy.stats import beta
	import matplotlib.pyplot as plt
	
	# a)
	mean = 0.2
	var = 0.01
	a = mean*((mean*(1-mean)/var)-1)
	b = a*(1-mean)/mean
	
	x = np.arange(0.0, 1.0, 0.001)
	y = beta.pdf(x, a, b)
	
	plt.figure()
	plt.plot(x, y)
	plt.xlabel('x')
	plt.ylabel('Beta(x,a,b)')
	plt.title("density function of Beta-distribution")
	plt.grid(True)
	
	
	# b)
	samples = beta.rvs(a, b, size=1000)
	plt.figure()
	plt.hist(samples, 20)
	plt.xlim([0, 1])
	plt.xlabel('x')
	plt.ylabel('frequency')
	plt.title("histogram of 1000 samples")
	
	
	# c)
	sample_mean = np.mean(samples)
	sample_var = np.var(samples, ddof=1)
	beta_mean, beta_var = beta.stats(a, b, moments="mv")
	# beta_mean = a/(a+b)
	# beta_var = (a*b)/(((a+b)**2)*(a+b+1))
	print("sample mean = %.6f, "
	"sample variance = %.6f"
	% (sample_mean, sample_var))
	print("true mean = %.6f, "
	"true variance = %.6f"
	% (beta_mean, beta_var))
	
	# d)
	cp95_interval = np.percentile(samples, [2.5, 97.5])
	print("The central 95%-intercal:" 
	" {}".format(cp95_interval))
	
	plt.show()
	
	\end{lstlisting}
\end{document} 
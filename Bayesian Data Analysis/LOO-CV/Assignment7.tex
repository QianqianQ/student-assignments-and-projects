\documentclass[article,11pt]{article}
\usepackage{listingsutf8}
\usepackage{float}
\usepackage{caption}
\usepackage{graphicx}
\usepackage{amsmath}
\usepackage{amssymb}
\usepackage{pythonhighlight}
\usepackage{listings}
\usepackage{verbatim}
\usepackage{enumitem}   
\title{\huge Bayesian Data Analysis - Assignment 7}
\begin{document}
 \maketitle
\begin{table}[h]
	\centering
	\begin{tabular}{|c|c|c|c|}
		\hline
		\textbf{Model Type} & \textbf{PSIS-LOO} & \textbf{$\bf{p_{eff}}$} & \textbf{k-value($\bf{k>0.7})$)}   \\ \hline\hline
		Pooled  & -131.0168 &2.042215  &  0 (0.0\%)    \\ \hline
		Separate & -132.8809 & 10.13113 &   4 (13.3\%)  \\ \hline
		Hierarchical &-126.6512  & 5.536214  &  0(0.0\%) \\ \hline
	\end{tabular}
	\label{tab:t1}
\end{table}
\begin{figure}[H]
	\centering
	\captionsetup{justification=centering}
	\includegraphics[width=15cm]{3_kvalues.jpeg}
	\caption{histogram of the k-values}
\end{figure}
\noindent{From the above table and figure, we can get the following conclusion: }\\ \\
1. All the k-values of the pooled model and those of the hierarchical model are less than or equal to 0.7. So the PSIS-LOO estimate of the pooled model and hierarchical model can be considered to be reliable.\\ \\
2. There are 13.3 of the k-values are more than 0.7 using separate model. So the PSIS-LOO estimate of the separate model can not be considered to be reliable. \\ \\ 
3. Considering the differences of the PSIS-LOO value between the three model as well, hierachical model should be selected. Because it has the highest PSIS-LOO value, meaning that it fits the data best, and it is a reliable model as we discuss before.
\\
\\
\appendix
\leftline{\LARGE \textbf{Appendix}}
\section{Stan}
\subsection{Pooled model}
\verbatiminput{pooled_factory.stan}
\subsection{Separate model}
\verbatiminput{separate_factory.stan}
\subsection{Hierachical model}
\verbatiminput{hierarchical_factory.stan}	
\section{R code}
\verbatiminput{factory.R}
\end{document}
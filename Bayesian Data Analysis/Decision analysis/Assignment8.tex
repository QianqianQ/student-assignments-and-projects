\documentclass[article,11pt]{article}
\usepackage{listingsutf8}
\usepackage{float}
\usepackage{caption}
\usepackage{graphicx}
\usepackage{amsmath}
\usepackage{amssymb}
\usepackage{pythonhighlight}
\usepackage{listings}
\usepackage{verbatim}
\usepackage{enumitem}   
\title{\huge Bayesian Data Analysis - Assignment 8}
\begin{document}
 \maketitle
 \section*{Decision analysis for the factory data}
 From the known information, we can get the utility function:
 \begin{equation*}
 U=
 \begin{cases}
 100& q \ge 85\\
 -100& q<85 
 \end{cases} 
 \end{equation*}
 
 \begin{equation*}
  E(U) = 100P(q\ge 85)-100P(q<85)
 \end{equation*}
 \begin{table}[H]
 	\centering
 	\begin{tabular}{|c|c|}
 		\hline
 		\textbf{Machine Number} & \textbf{expected utility(euros)}    \\ \hline\hline
 		4  & 81.75   \\ \hline
 		2 & 72.40   \\ \hline
 		5 &29.15  \\ \hline
 		3 &20.90  \\ \hline
 		6 &15.60  \\ \hline
 		1 &-22.40  \\ \hline
 	\end{tabular}
 	\caption{Ranking of the expected utilities}
 	\label{tab:t1}
 \end{table}
\noindent{From the Table 1, we can get the following conclusion:}\\
	1. The 4th machine can make the most profit, meaning that its quality is best. \\
	2. Besides the 4th machine, the quality of the 2nd machine is also great. \\
	3. The 5th, 3rd, 6th machine make profits which are not high. Their qualities are acceptable. \\
	4. The only one machine which causes loss is the 1st machine. It has the worst quality and it is not profitable. This machine should be discarded or repaired.\\ \\
	
	\noindent{The expected utility of the 7th machine is 28. It is profitable, even though the expected utility is not so high. So buying a new(7th) machine is recommended.} \\ \\
	
	\appendix
	\leftline{\LARGE \textbf{Appendix}}
	\section{Stan}
	\verbatiminput{hierarchical_factory.stan}
	\section{R code}
	\verbatiminput{factory.R}
\end{document}